\documentclass{article}
\usepackage{graphics}
\title{CHESS project}
\author{Kunal kumar and  Karthik VV}
\begin{document}

\maketitle

\begin{figure}[ht]
\centering
\includegraphics{chess.jpeg}
\caption{{\Large chess board}}
\label{fig1}
\end{figure}



\section{Our project: }
The objective of our project is to create an online Chess game which enables you to both play with your friends as well as against the computer.\\



\section{What is Chess?}
Chess is a two-player strategy board game played on a chessboard, a checkered gameboard with 64 squares arranged in an 8×8 grid. The game is played by millions of people worldwide. Chess is believed to have originated in India sometime before the 7th century. The game was derived from the Indian game chaturanga, which is also the likely ancestor of the Eastern strategy games xiangqi, janggi, and shogi. Chess reached Europe by the 9th century, due to the Umayyad conquest of Hispania. The pieces assumed their current powers in Spain in the late 15th century.
\subsection{Each player starts with 16 pieces namely:}
\begin{itemize}
\item {one King}
\item {one Queen}
\item {two Rooks}
\item {two Knights}
\item {two Bishops}
\item {eight Pawns}
\end{itemize}

\subsection{Moves:}
\begin{enumerate}
\item {The king moves one square in any direction. The king also has a special move called castling that involves also moving a rook.}
\item {The queen can move any number of squares along a rank, file, or diagonal, but cannot leap over other pieces.}
\item {The rook can move any number of squares along a rank or file, but cannot leap over other pieces. Along with the king, a rook is involved during the king's castling move.}
\item {The bishop can move any number of squares diagonally, but cannot leap over other pieces.}
\item {The knight moves in "L"-shape. The knight is the only piece that can leap over other pieces.}
\item {The pawn can move one step ahead, or on its first move it can advance two squares, provided they are unoccupied; or the pawn can capture an opponent's piece on a square diagonally in front of it. A pawn has two special moves: the en passant capture and promotion.}
\end{enumerate}


\subsection{Rules:}
\begin{itemize}
	\item {Player with white pieces starts his turn first.}
	\item {Player has to protect his king and subsequently make strategies to kill the opponent's king. }
	\item {When the king of one player is checked by the other player's pieces, he will have to take the king out of that check to a safe place. The player who checks the king has to warn the opponent by saying "CHECK". }
	\item {The one to lose his king will be the one to lose the game.}
\end{itemize}



\section{Features:}
\subsection{Player vs Computer:}
The player will have to start with whites and make your first move, and the computer will take blacks.

\subsection{Player vs Player:}
The player who takes whites should start first and the other player plays alternately.\\



\section{Main Functions in Player vs Computer code:}
\subsection{updateSquarecolor()}
This function is used to update the colors of each square after each move.
\subsection{checkBlack(n,values)}
This function is used to find all the possible places to where a selected white chess piece can be moved in this step.
\subsection{checkWhite(n,values)}
This function is used to find all the possible places to where a selected black chess piece can be moved in this step.
\subsection{checkmate()}
This function is used to check if there would be danger to your king if you make any move and declares you to have lost.
\subsection{check()}
This function is used to make your moves, if possible, and also to warn the player if the mkintg is in check and the player is not trying to protect it, by the use of the checkBlack and checkWhite functions described above.
\subsection{chooseTurn()}
This function is helpful for the computer to play its move by moving forward with the best possible move. It also declares the player as winner if there is no possible move for the computer to make while protecting its king. It basically helps in the propogation of the game step by step.
\subsection{startTime()}
This function is to show current time and day.
\subsection{maxxTime()}This function is used to check the time limit for each move and warn the player after 45 seconds and stop the game and declare him to have lost after 1 minute.
\subsection{maxxTime2()}This function is used to check and stop the game at maximum time limit of 45 minutes and the game would be declared drawn.
\subsection{restart()}This function is used to restart timer after each move. \\





\section{Main functions in Player vs Player code:}
\subsection{updateSquarecolor()}This function is used to update the colors of each square after each move.
\subsection{checkBlack(n,values)}This function is used to find all the possible places to where a selected white chess piece can be moved in this step.
\subsection{checkWhite(n,values)}This function is used to find all the possible places to where a selected black chess piece can be moved in this step.
\subsection{checkmate()}This function is used to check if there would be danger to one's king if they make any move and declares him to have lost.
\subsection{check()}This function is used to make your moves, if possible, and also to warn the player if the mkintg is in check and the player is not trying to protect it, by the use of the checkBlack and checkWhite functions described above.
\subsection{startTime()}This function is to show current time and day.
\subsection{maxxTime()}This function is used to check the time limit for each move and warn the player after 45 seconds and stop the game and declare him to have lost after 1 minute.
\subsection{maxxTime2()}This function is used to check and stop the game at maximum time limit of 45 minutes and the game would be declared drawn.
\subsection{restart()}This function is used to restart timer after each move.\\


\section{Installation:}
Download all our files and put them collectively in a folder named \textbf{"Chess".}

\subsection{In Ubuntu:} Now put this folder into the /var/www/html/ folder in the other locations on your computer.

\subsection{In Window:} Download XAMPP on your PC and put this folder into the htdocs folder of XAMPP.\\

Now go to your web browser and type localhost/Chess/\\

Best view in Google Chrome and Mozilla.

\begin{center}
\textbf{The End}
\end{center}
\end{document}